%%%%%%%%%%%%%%%%%%%%%%%%%%%%%%%%%%%%%%%%%%%%%%%%%%%%%%%%%%%%%%%%%%%%%%%%%%%%%%%%
% Medium Length Graduate Curriculum Vitae
% LaTeX Template
% Version 1.2 (3/28/15)
%
% This template has been downloaded from:
% http://www.LaTeXTemplates.com
%
% Original author:
% Rensselaer Polytechnic Institute 
% (http://www.rpi.edu/dept/arc/training/latex/resumes/)
%
% Modified by:
% Daniel L Marks <xleafr@gmail.com> 3/28/2015
% 
% Further modified by:
% Rohan Bavishi <rohan.bavishi95@gmail.com> 9/20/2016
%
% Further modified by:
% Nicolas Badoux <n.badoux@hotmail.com> 11/3/2018
%
% Important note:
% This template requires the simple_style.cls file to be in the same directory 
% as the .tex file. The res.cls file provides the resume style used for 
% structuring the document.
%
%%%%%%%%%%%%%%%%%%%%%%%%%%%%%%%%%%%%%%%%%%%%%%%%%%%%%%%%%%%%%%%%%%%%%%%%%%%%%%%%

%-------------------------------------------------------------------------------
%	PACKAGES AND OTHER DOCUMENT CONFIGURATIONS
%-------------------------------------------------------------------------------

%%%%%%%%%%%%%%%%%%%%%%%%%%%%%%%%%%%%%%%%%%%%%%%%%%%%%%%%%%%%%%%%%%%%%%%%%%%%%%%%
% You can have multiple style options the legal options ones are:
%
%   centered:	the name and address are centered at the top of the page 
%				(default)
%
%   line:		the name is the left with a horizontal line then the address to
%				the right
%
%   overlapped:	the section titles overlap the body text (default)
%
%   margin:		the section titles are to the left of the body text
%		
%   11pt:		use 11 point fonts instead of 10 point fonts
%
%   12pt:		use 12 point fonts instead of 10 point fonts
%
%%%%%%%%%%%%%%%%%%%%%%%%%%%%%%%%%%%%%%%%%%%%%%%%%%%%%%%%%%%%%%%%%%%%%%%%%%%%%%%%

\documentclass[mm, 11pt]{simple_style}  

% Default font is the helvetica postscript font
\usepackage{helvet}
\usepackage[hidelinks]{hyperref}
\usepackage{url}
\usepackage[a4paper, left=12mm, right=45mm, top=15mm, bottom=15mm]{geometry}
\usepackage[fristpage=true, color=black, opacity=0.5, angle=90]{background}

\begin{document}

\SetBgContents{As of \today. Latest: bit.ly/cvNicolas}% Set contents
\SetBgPosition{ \textwidth / 0.8 - 0.8cm,-23cm/0.8}% Select location
\SetBgScale{0.8}% Select scale factor of logo

\newsectionwidth{26mm}
%-------------------------------------------------------------------------------
%	NAME AND ADDRESS SECTION
%-------------------------------------------------------------------------------
\name{Nicolas Badoux}
%\qualification{Senior undergraduate ETHZ student}
\emailone{n.badoux@hotmail.com}
\emailtwo{nicolas.badoux@epfl.ch}
%\website{https://nicolasbadoux.com}{\url{nicolasbadoux.com}}
%\github{https://github.com/vwvw}{\url{github.com/vwvw}}
\phone{+41 79 914 00 47}
\linkedin{nbadoux}
\citizenship{Swiss citizen - married}
\birthdate{Born 06.11.1994}

\address{\href{https://maps.app.goo.gl/kmh2wtaNsmxrcqGc8}{Rue de la Gare 21}\\
         \href{https://maps.app.goo.gl/kmh2wtaNsmxrcqGc8}{1030} 
         \href{https://maps.app.goo.gl/kmh2wtaNsmxrcqGc8}{Bussigny}\\
         \href{https://maps.app.goo.gl/kmh2wtaNsmxrcqGc8}{CH - Switzerland}
}

\begin{resume}
%-------------------------------------------------------------------------------
\section{Education}
\cusemph{PhD candidate in Computer Science}
\timeline{2020-\textit{2025}}\\
\textsl{\href{https://ic.epfl.ch/en}{\'Ecole Polytechnique F\'ed\'erale de Lausanne (EPFL)}} - Switzerland\\
- Supervisor: \href{https://nebelwelt.net/}{Prof. Mathias Payer} @ \href{https://hexhive.epfl.ch}{HexHive}\\
- Topic: Compiler-based defenses and testing, system security\\
%-------------------------------------------------------------------------------
\cusemph{Master of Science ETH in Computer Science}
\timeline{2016-2019}\\
\textsl{\href{https://www.inf.ethz.ch/}{Eidgen\"ossische Technische Hochschule Z\"urich (ETHZ)}} - Switzerland\\
- Specialization in Information Security, GPA: 5.39/6\\
%-------------------------------------------------------------------------------
\cusemph{Bachelor in Communication Sciences}
\timeline{2013-2016}\\
\textsl{\href{https://ic.epfl.ch/en}{\'Ecole Polytechnique F\'ed\'erale de Lausanne (EPFL)}} - Switzerland\\
- Exchange program @
\cusemph{\href{https://www.ece.cmu.edu/}{Carnegie Mellon University}} - USA, GPA: 5.26/6
\timeline{2015-2016}\\
\sectionline
%-------------------------------------------------------------------------------
%-------------------------------------------------------------------------------
%	RESEARCH SECTION
%-------------------------------------------------------------------------------
\section{Research Experience}
\begin{research}
    \title{\texttt{type++}: prohibiting type confusion with inline type information}
    \location{NDSS'25}
    \description{%
      \item By inlining the type in each C++ object, we create a compiler-based
      mitigation against type confusion attacks allowing derived cast to be checked at
      runtime while requiring minimal code adaptations. We evaluate our prototype against the state-of-the-art.
      \item Built on top of LLVM, \texttt{type++} protects from
      type confusions with less than 3\% overhead.
    }
\end{research}
\begin{research}
    \title{\textsc{libErator}: Balancing library fuzzing without consumer code}
    \location{Submitted @ FSE'25}
    \description{%
      \item We automate the generation of library fuzzing drivers and allow for balancing resources between driver generation and fuzzing.
      \item From insights gathered through LLVM passes, we build valid C drivers using API functions. 
    }
\end{research}
\begin{research}
    \title{Bypassing LLVM-CFI cast protection }
    \location{Ongoing}
    \description{
      \item We present a novvel attack against LLVM-CFI, bypassing the cast protection for C++.
    }
\end{research}
\vspace{-\parskip}
\sectionline
%-------------------------------------------------------------------------------
%  SKILLS SECTION
%-------------------------------------------------------------------------------
\section{Skills}
\cusemph{Programming Languages}: Python, C++, \LaTeX, Bash\\
\cusemph{Software}: LLVM, GDB, libfuzzer, Linux, Docker\\
\cusemph{Spoken Languages}: French (native), English, Swiss-German, German\\
\sectionline
%-------------------------------------------------------------------------------
%-------------------------------------------------------------------------------
%	WORK SECTION
%-------------------------------------------------------------------------------
\section{Industry Experience}
\begin{position}
    \employer{\href{https://digger.ch}{Fondation Digger}\normalfont{, NGO}}
    \title{Software Engineer}
    \location{Tavannes, CH}
    \duration{Aug' 2019 - March 2020}
    \description{
      \item Developed a virtual overlay for remotely removing explosives with the help of \href{https://opencv.org}{OpenCV} and \href{https://unity.com/}{Unity} in an Agile environment as part of \href{https://en.wikipedia.org/wiki/Swiss_Civilian_Service}{mandatory civil service}.
    }
\end{position}
\begin{position}
    \employer{\href{https://compassion.ch}{Compassion Suisse}\normalfont{, NGO}}
    \title{Software Engineer}
    \location{Yverdon, CH}
    \duration{Mar' - May 2018}
    \description{
        \item As part of my
        \href{https://en.wikipedia.org/wiki/Swiss_Civilian_Service}{mandatory
        Swiss civil Service}, contributed to
        \href{https://github.com/CompassionCH/compassion-switzerland}{opensource}
        Python modules for the \href{https://www.odoo.com/}{Odoo ERP}.
    }
\end{position}
\begin{position}
    \employer{\href{https://ergon.ch/en}{Ergon Informatik}}
    \title{Security Engineer Intern}
    \location{Z\"urich, CH}
    \duration{60\% - Sept' 2017 - Mar' 2018}
    \description{
        \item Developed a blackbox fuzzer in Python to find bugs in Ergon's \href{https://www.airlock.com/en/products/airlock-waf/}{Web Application Firewall}.
    }
\end{position}
\begin{position}
    \employer{\href{https://www.morganstanley.com/}{Morgan Stanley}}
    \title{Technology Summer Analyst}
    \location{London, UK}
    \duration{June - Aug' 2016}
    \description{
        \item Developed charts in AngularJS for statistics of the Architecture Security team.
    }
\end{position}
\vspace{-\parskip}
\sectionline
%-------------------------------------------------------------------------------
%  PROJECTS SECTION
%-------------------------------------------------------------------------------
%\section{Projects}
%%
%\begin{project}
%  \title{MinFuzz: Program simplification to drive fuzzing effectiveness}
%  \duration{Winter 2018}
%  \location{EPFL}
%  \description{
%    \item Development of automatic code pruning targeting checksum, decryption, and decompression functions to maximize number of execution and code coverage exercised.
%
%    \item Resulting in a 50\% speed up of targeted binaries and coverage up to 20\% higher than AFL.
%  }
%\end{project}
%%
%\begin{project}
%  \title{\href{https://www.ethz.ch/en/the-eth-zurich/global/global-network/eth-studios/new-york-security-challenge-2018.html}{Bloomberg New York Security Challenge}}
%  \duration{June 2018}
%  \location{New York, USA}
%  \description{
%    \item As a winner of
%    \href{https://blogs.ethz.ch/ETHambassadors/2018/07/26/coding-challenge-inside-bloomberg/}{ETH
%    Codecon}, participated in a week-long project at Bloomberg
%    \href{https://www.bloomberg.com/careers/technology/}{CTO Office}, analyzing
%    network traces of microservices for security anomalies.
%  }
%\end{project}
%%
%\begin{project}
%  \title{\href{https://github.com/vwvw/Ethereum-RingCT}{RingCT in an Ethereum smart contract}}
%  \duration{Mar' - Sept' 2017}
%  \location{ETHZ}
%  \description{
%	\item Implemented in Solidity a smart contract executing Monero-like confidential transactions.\par
%	\item Part of an introduction to Research under the supervision of \href{https://arthurgervais.com/}{Prof. Arthur Gervais}.
%  }
%\end{project}
%\sectionline
%-------------------------------------------------------------------------------
%  TEACHING SECTION
%-------------------------------------------------------------------------------
\section{Teaching Assistant}
%

{
\cusemph{CS-119 - Information, Calcul \& Communication} \hfill\textit{ 2022 \& 2024}
  \parskip 0pt

\cusemph{CS-323 - Operating System} \hfill \textit{2021}

\cusemph{CS-412 - Software Security} \hfill \textit{2021 \& 2023}

\cusemph{COM-402 - Information Security \& Privacy} \hfill \textit{2023}
}\\
\sectionline
%\section{Talks}
%
%\cusemph{NDSS'25} - \texttt{type++}: prohibiting type confusion with inline type information \timeline{2025}
%
%\sectionline
%-------------------------------------------------------------------------------

%-------------------------------------------------------------------------------
%	Interests
%-------------------------------------------------------------------------------
\section{Activities}
\begin{research}
  \title{Board Member, Treasurer \textnormal{- \href{https://www.gbeu.ch}{Groupes Bibliques des Écoles et Universités}}}
  \location{2023 - present}
  \description{%
    \item Define the vision, hiring of the general secretary, and budget planning ($\simeq$ 500kCHF).
  }
\end{research}
\begin{research}
  \title{Camp Leader \textnormal{- \href{https://www.interjeunes.net/explosion-camp}{Interjeunes} \& \href{https://ligue.ch}{Ligue pour la Lecture de la Bible}}}
  \location{2014, 2017, 2021, 2022}
  \description{
    \item Lead camps with up 110 kids/young adults for a week. Built a team, prepared the event, managed the team and was the authority in charge during the week.
  }
\end{research}


\end{resume}
\end{document}
