%%%%%%%%%%%%%%%%%%%%%%%%%%%%%%%%%%%%%%%%%%%%%%%%%%%%%%%%%%%%%%%%%%%%%%%%%%%%%%%%
% Medium Length Graduate Curriculum Vitae
% LaTeX Template
% Version 1.2 (3/28/15)
%
% This template has been downloaded from:
% http://www.LaTeXTemplates.com
%
% Original author:
% Rensselaer Polytechnic Institute 
% (http://www.rpi.edu/dept/arc/training/latex/resumes/)
%
% Modified by:
% Daniel L Marks <xleafr@gmail.com> 3/28/2015
% 
% Further modified by:
% Rohan Bavishi <rohan.bavishi95@gmail.com> 9/20/2016
%
% Further modified by:
% Nicolas Badoux <n.badoux@hotmail.com> 11/3/2018
%
% Important note:
% This template requires the simple_style.cls file to be in the same directory 
% as the .tex file. The res.cls file provides the resume style used for 
% structuring the document.
%
%%%%%%%%%%%%%%%%%%%%%%%%%%%%%%%%%%%%%%%%%%%%%%%%%%%%%%%%%%%%%%%%%%%%%%%%%%%%%%%%

%-------------------------------------------------------------------------------
%	PACKAGES AND OTHER DOCUMENT CONFIGURATIONS
%-------------------------------------------------------------------------------

%%%%%%%%%%%%%%%%%%%%%%%%%%%%%%%%%%%%%%%%%%%%%%%%%%%%%%%%%%%%%%%%%%%%%%%%%%%%%%%%
% You can have multiple style options the legal options ones are:
%
%   centered:	the name and address are centered at the top of the page 
%				(default)
%
%   line:		the name is the left with a horizontal line then the address to
%				the right
%
%   overlapped:	the section titles overlap the body text (default)
%
%   margin:		the section titles are to the left of the body text
%		
%   11pt:		use 11 point fonts instead of 10 point fonts
%
%   12pt:		use 12 point fonts instead of 10 point fonts
%
%%%%%%%%%%%%%%%%%%%%%%%%%%%%%%%%%%%%%%%%%%%%%%%%%%%%%%%%%%%%%%%%%%%%%%%%%%%%%%%%

\documentclass[mm, 11pt]{simple_style}  

% Default font is the helvetica postscript font
\usepackage{helvet}
\usepackage{hyperref}
\usepackage{url}
\usepackage{rotating}
\usepackage{xcolor}
\hypersetup {
    colorlinks=false,
    linkcolor=colorlink,
    filecolor=magenta,      
    urlcolor=colorlink,
}
\usepackage[a4paper, left=12mm, right=45mm, top=15mm, bottom=15mm]{geometry}
\usepackage[fristpage=true, color=black, opacity=0.5, angle=90]{background}

\begin{document}

\SetBgContents{As of \today. Latest: bit.ly/cvNicolas}% Set contents
\SetBgPosition{ \textwidth / 0.8 - 0.8cm,-23cm/0.8}% Select location
\SetBgScale{0.8}% Select scale factor of logo

\newsectionwidth{26mm}
%-------------------------------------------------------------------------------
%	NAME AND ADDRESS SECTION
%-------------------------------------------------------------------------------
\name{Nicolas Badoux}
%\qualification{Senior undergraduate ETHZ student}
\emailone{n.badoux@hotmail.com}
%\emailtwo{badouxn@ethz.ch}
%\website{https://nicolasbadoux.com}{\url{nicolasbadoux.com}}
%\github{https://github.com/vwvw}{\url{github.com/vwvw}}
\phone{+41 79 914 00 47}
\citizenship{Swiss citizen}
\birthdate{Born 06.11.1994}

\address{\href{https://goo.gl/maps/bxwsN3DiRHF2}{Grand-Rue 8}
    \\\href{https://goo.gl/maps/bxwsN3DiRHF2}{1315} \href{https://goo.gl/maps/bxwsN3DiRHF2}{La Sarraz}
    \\\href{https://goo.gl/maps/bxwsN3DiRHF2}{CH - Switzerland}
}

\begin{resume}
%-------------------------------------------------------------------------------
\section{Education}
\cusemph{\href{https://ic.epfl.ch/en}{\'Ecole Polytechnique F\'ed\'erale de Lausanne (EPFL)}} - Switzerland  \timeline{2020-ongoing}\\
{\sl PhD student in Computer Science} in the \href{hexhive.epfl.ch}{Hexhive} group\\
Under supervision from \href{nebelwelt.net}{Prof. Mathias Payer}\\
\cusemph{\href{https://www.inf.ethz.ch/}{Eidgen\"ossische Technische Hochschule Z\"urich (ETHZ)}} - Switzerland \timeline{2016-2019}\\
{\sl Master of Science ETH in Computer Science}\\
Specialization in Information Security\\
\cusemph{\href{https://ic.epfl.ch/en}{\'Ecole Polytechnique F\'ed\'erale de Lausanne (EPFL)}} - Switzerland  \timeline{2013-2016}\\
{\sl Bachelor in Communication Sciences}, GPA: 5.26 / 6\\
\cusemph{\href{https://www.ece.cmu.edu/}{Carnegie Mellon University (CMU)}} - Pittsburgh PA, USA \timeline{2015-2016}\\
Two semester exchange program\\
\sectionline
%-------------------------------------------------------------------------------
%-------------------------------------------------------------------------------
%	WORK SECTION
%-------------------------------------------------------------------------------
\section{Work Experience}
\begin{position}
    \employer{\href{https://digger.ch}{Fondation Digger}\normalfont{, NGO}}
    \title{Software Engineer}
    \location{Tavannes, CH}
    \duration{Aug 2019 - March 2020}
    \description{-\,Development of an virtual overlay for a video stream to be stream in VR  for demining work.\par
    -\,Extensive use of \href{https://opencv.org}{OpenCV} and \href{https://unity.com/}{Unity} in an Agile environment for my \href{https://en.wikipedia.org/wiki/Swiss_Civilian_Service}{mandatory civil Service}}
\end{position}\\
\begin{position}
    \employer{\href{https://chainsolutions.com/}{chainSolutions}}
    \title{Blockchain Consultant}
    \location{Z\"urich, CH}
    \duration{Jan' - June 2018 @ 20\%}
    \description{
        -Undertook the conception of the technical foundations for different clients project.\par
        -Designed part of a global self-sovereign identity management system.}
\end{position}\\
\begin{position}
    \employer{\href{https://compassion.ch}{Compassion Suisse}\normalfont{, NGO}}
    \title{Software Engineer}
    \location{Yverdon, CH}
    \duration{Mar' - May 2018}
    \description{
        -Contributed to  \href{https://github.com/CompassionCH/compassion-switzerland}{opensource} modules for the \href{https://www.odoo.com/}{Odoo} ERP. Written in Python in an Agile team as part of my \href{https://en.wikipedia.org/wiki/Swiss_Civilian_Service}{mandatory Swiss civil Service}
    }
\end{position}\\
\begin{position}
    \employer{\href{https://ergon.ch/en}{Ergon Informatik}}
    \title{Security Engineer Intern}
    \location{Z\"urich, CH}
    \duration{Sept' 2017 - Mar' 2018 @ 60\%}
    \description{
        -Development in Python of a blackbox fuzzer to improve security of Ergon \href{https://www.airlock.com/en/products/airlock-waf/}{Web Application Firewall}, Airlock.\par
        -Use of \href{https://gitlab.com/akihe/radamsa}{Radamsa} to generate malformed HTTP multipart requests resulting in the correction of multiple logic bugs.
    }
\end{position}\\
\begin{position}
    \employer{\href{https://www.morganstanley.com/}{Morgan Stanley}}
    \title{Technology Summer Analyst}
    \location{London, UK}
    \duration{June - Aug' 2016}
    \description{
        -Developed customizable charts providing visualization of the statistics of the Architecture Security team.\par
        -Identified the important charts and use cases. Implemented charts in AngularJS with a Java EE backend.
    }
\end{position}\\
\sectionline
\section{Relevant courses}
    \begin{tabular}{lllll}
        \href{http://www.syssec.ethz.ch/education/sown/sown_AS16.html}{Security of Wireless Network} & \href{http://www.syssec.ethz.ch/education/system_security/system_security_as16.html}{System Security}\\
        \href{https://netsec.ethz.ch/courses/netsec-2016/}{Network Security} & \href{http://www.infsec.ethz.ch/education/as2016.html}{Applied Security Lab} \\
        \href{https://www.cs.cmu.edu/~10601b/}{Introduction to Machine Learning} & \href{https://ndal.ethz.ch/courses/acn.html}{Advanced Computer Networks}\\
    \end{tabular}\\
\sectionline
%-------------------------------------------------------------------------------
%  SKILLS SECTION
%-------------------------------------------------------------------------------
\section{Skills}
\cusemph{Programming Languages}: Python, LLVM, Java, Matlab, Solidity, \LaTeX\\
\cusemph{Spoken Languages}: French (native), English, Swiss-German, German\\
\sectionline
%-------------------------------------------------------------------------------
%  PROJECTS SECTION
%-------------------------------------------------------------------------------
\section{Projects}
%
\begin{project}
  \title{MinFuzz: Program simplification to drive fuzzing effectiveness}
  \duration{Winter 2018}
  \location{EPFL}
  \description{
    -Development of automatic code pruning targeting checksum, decryption, and decompression functions to maximize number of execution and code coverage exercised.\par
    -Resulting in a 50\% speed up of targeted binaries and coverage up to 20\% higher than AFL.\par
	-6 months Master Thesis written in Python under the supervision of \href{https://nebelwelt.net/}{Prof. Mathias Payer}
  }
\end{project}
%
\begin{project}
  \title{\href{https://www.ethz.ch/en/the-eth-zurich/global/global-network/eth-studios/new-york-security-challenge-2018.html}{Bloomberg New York Security Challenge}}
  \duration{June 2018}
  \location{New York, USA}
  \description{
    -As a winner of \href{https://blogs.ethz.ch/ETHambassadors/2018/07/26/coding-challenge-inside-bloomberg/}{ETH Codecon}, participated in a week-long project at Bloomberg \href{https://www.bloomberg.com/careers/technology/}{CTO Office}, analyzing network traces of microservices for security anomalies.\\
  }\\
\end{project}
%
\begin{project}
  \title{\href{https://github.com/vwvw/Ethereum-RingCT}{RingCT in an Ethereum smart contract}}
  \duration{Mar' - Sept' 2017}
  \location{ETHZ}
  \description{
	-Implemented in Solidity a smart contract executing Monero-like confidential transactions.\par
	-Part of an introduction to Research under the supervision of \href{http://arthurgervais.com/}{Prof. Arthur Gervais}.\\
  }\\
\end{project}\\
%\begin{project}
%  \title{\href{https://github.com/vwvw/Greenheater}{Greenheater - a smart electric heater}}
%  \duration{Jan' - May 2016}
%  \location{CMU}
%  \description{
%	-Automated a personal space heater to take advantage of the periods where the user is not in the room to save electricity without impacting his comfort.Written mostly in Python.\\
%  }\\
%\end{project}\\
% \begin{project}
%   \title{Transpondeur sans fil sur transmission hertzienne}
%   \duration{Aug' - Dec' 2015}
%   \location{CMU}
%   \description{
% 	-Programmation d’un syst\`eme de communication digital capable d’envoyer plusieurs dizaines de milliers de bits par seconde sans erreur.\\
% 	\hspace{-1.1ex}-\'Ecrit en Matlab dans le but de d\'ecouvrir diff\'erents principes de communication digitale.\\
%   }\\
% \end{project}\\
\sectionline
%-------------------------------------------------------------------------------

%-------------------------------------------------------------------------------
%	Interests
%-------------------------------------------------------------------------------
\section{Activities}
\cusemph{\href{https://www.interjeunes.net/explosion-camp/}{Winter Camp Leader}} - Grimentz, CH \timeline{2014 \& 2017}\\
Camp with 110 teenagers/young adults during a week. Built a team, prepared the event, managed the team and was in charge of the authority during the week.\\
\end{resume}
\end{document}