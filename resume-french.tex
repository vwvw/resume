%%%%%%%%%%%%%%%%%%%%%%%%%%%%%%%%%%%%%%%%%%%%%%%%%%%%%%%%%%%%%%%%%%%%%%%%%%%%%%%%
% Medium Length Graduate Curriculum Vitae
% LaTeX Template
% Version 1.2 (3/28/15)
%
% This template has been downloaded from:
% http://www.LaTeXTemplates.com
%
% Original author:
% Rensselaer Polytechnic Institute 
% (http://www.rpi.edu/dept/arc/training/latex/resumes/)
%
% Modified by:
% Daniel L Marks <xleafr@gmail.com> 3/28/2015
% 
% Further modified by:
% Rohan Bavishi <rohan.bavishi95@gmail.com> 9/20/2016
%
% Further modified by:
% Nicolas Badoux <n.badoux@hotmail.com> 11/3/2018
%
% Important note:
% This template requires the simple_style.cls file to be in the same directory 
% as the .tex file. The res.cls file provides the resume style used for 
% structuring the document.
%
%%%%%%%%%%%%%%%%%%%%%%%%%%%%%%%%%%%%%%%%%%%%%%%%%%%%%%%%%%%%%%%%%%%%%%%%%%%%%%%%

%-------------------------------------------------------------------------------
%	PACKAGES AND OTHER DOCUMENT CONFIGURATIONS
%-------------------------------------------------------------------------------

%%%%%%%%%%%%%%%%%%%%%%%%%%%%%%%%%%%%%%%%%%%%%%%%%%%%%%%%%%%%%%%%%%%%%%%%%%%%%%%%
% You can have multiple style options the legal options ones are:
%
%   centered:	the name and address are centered at the top of the page 
%				(default)
%
%   line:		the name is the left with a horizontal line then the address to
%				the right
%
%   overlapped:	the section titles overlap the body text (default)
%
%   margin:		the section titles are to the left of the body text
%		
%   11pt:		use 11 point fonts instead of 10 point fonts
%
%   12pt:		use 12 point fonts instead of 10 point fonts
%
%%%%%%%%%%%%%%%%%%%%%%%%%%%%%%%%%%%%%%%%%%%%%%%%%%%%%%%%%%%%%%%%%%%%%%%%%%%%%%%%

\documentclass[mm, 11pt]{simple_style}  

% Default font is the helvetica postscript font
\usepackage{helvet}
\usepackage[hidelinks]{hyperref}
\usepackage{url}
\usepackage[a4paper, left=12mm, right=45mm, top=15mm, bottom=15mm]{geometry}
\usepackage[fristpage=true, color=black, opacity=0.5, angle=90]{background}
\usepackage[ddmmyyyy]{datetime}
\renewcommand{\dateseparator}{.}
\renewcommand{\authorsText}{Auteurs}
\fancyfoot[L]{\hspace{-3cm}\footnotesize{\textcolor{gray}{En date du \today. Version actuelle: \href{https://nicolasbadoux.com/cv}{nicolasbadoux.com/cv}}}}% Set contents

\begin{document}

\newsectionwidth{29mm}
%-------------------------------------------------------------------------------
%	NAME AND ADDRESS SECTION
%-------------------------------------------------------------------------------
\name{Nicolas Badoux}
%\qualification{Senior undergraduate ETHZ student}
\emailone{n.badoux@hotmail.com}
\emailtwo{nicolas.badoux@epfl.ch}
%\website{https://nicolasbadoux.com}{\url{nicolasbadoux.com}}
%\github{https://github.com/vwvw}{\url{github.com/vwvw}}
\phone{+41 79 914 00 47}
\linkedin{nbadoux}
\citizenship{Suisse---marié}
\birthdate{Né le 06.11.1994}

\address{\href{https://maps.app.goo.gl/kmh2wtaNsmxrcqGc8}{Rue de la Gare 21}\\
         \href{https://maps.app.goo.gl/kmh2wtaNsmxrcqGc8}{1030} 
         \href{https://maps.app.goo.gl/kmh2wtaNsmxrcqGc8}{Bussigny}\\
         \href{https://maps.app.goo.gl/kmh2wtaNsmxrcqGc8}{CH---Switzerland}
}

\begin{resume}
%-------------------------------------------------------------------------------
\section{Formations}

\cusemph{Docteur ès sciences (PhD)}
\timeline{2020--\textit{2025}}\\
\textsl{\href{https://ic.epfl.ch/en}{\'Ecole Polytechnique F\'ed\'erale de Lausanne (EPFL)}} - Suisse
\begin{itemize}
  \item Directeur de thèse: \href{https://nebelwelt.net/}{Prof. Mathias Payer} au sein du laboratoire \href{https://hexhive.epfl.ch}{HexHive}.
  \item Thèse: Sécuriser le code bas niveau avec un minimum d'efforts pour les développeurs.
  \item Thèmes: Sécurité des systèmes, tests logiciels, protections par les compilateurs, fuzzing. 
\end{itemize}


%-------------------------------------------------------------------------------
\cusemph{Master of Science ETH en informatique}
\timeline{2016--2019}\\
\textsl{\href{https://www.inf.ethz.ch/}{Eidgen\"ossische Technische Hochschule Z\"urich (ETHZ)}} - Suisse

\begin{itemize}
  \item  Spécialisation en Sécurité informatique, Moyenne: 5.39/6.
\end{itemize}
%-------------------------------------------------------------------------------
\cusemph{Bachelor en Systèmes de Communication}
\timeline{2013--2016}\\
\textsl{\href{https://ic.epfl.ch/en}{\'Ecole Polytechnique F\'ed\'erale de Lausanne (EPFL)}} - Suisse

\begin{itemize}
  \item Année d'échange @
\cusemph{\href{https://www.ece.cmu.edu/}{Carnegie Mellon University}} - USA, Moyenne: 5.26/6.
\timeline{2015--2016\hspace{-0.2em}}
\end{itemize}
%-------------------------------------------------------------------------------
\cusemph{Maturité bilingue (Allemand/Français)}
\timeline{2010--2013}\\
\textsl{\href{https://www.kanti-frauenfeld.ch/}{Kantonschule Frauenfeld} \& Gymnase d'Yverdon} - Suisse

\begin{itemize}
  \item Option spécifique: Physique et application des mathématiques, Moyenne: 5.19/6, top 3\%.
\end{itemize}

\sectionline
%-------------------------------------------------------------------------------
%-------------------------------------------------------------------------------
%	RESEARCH SECTION
%-------------------------------------------------------------------------------
\section{\'Experience en recherche}
\begin{research}
    \title{\href{https://hexhive.epfl.ch/publications/files/25NDSS.pdf}{\texttt{type++}:
    prohibiting type confusion with inline type information}}
    \location{\href{https://www.ndss-symposium.org/ndss2025/}{NDSS'25}}
    \authors{\textbf{Nicolas Badoux}, Flavio Toffalini, Yuseok Jeon, \& Mathias Payer.}
    \description{%
      \item \textit{Distinction des meilleurs articles} (top 5\%).
      %
      \item En C++, un downcast incorrect peut mener à des vulnérabilités sévères.
      %
      \item En ajoutant un type à chaque objet C++, notre compilateur permet de
      vérifier chaque conversion. Prévenant tout risque
      de confusion de type à un nombre minime d'adaptations du code source.
      Nous obtenons moins de 1\% de ralentissement tout en protégeant 90
      milliards de conversions. Nous déployons notre prototype sur Chromium. Bâti
      sur LLVM, \texttt{type++} est disponible sur
      \href{https://github.com/HexHive/typepp}{\underline{GitHub}} et son
      artefact a été évalué.%
      %
      \item En tant que leader de ce projet long de plusieurs années, j'ai acquis des
      compétences techniques, rédactionnelles, et stratégiques, par exemple, sur l'articulation
      d'un projet dans un domaine en constante évolution.
    }
\end{research}
\begin{research}
    \title{\href{https://nebelwelt.net/files/25FSE2.pdf}{\textsc{libErator}: Balancing library fuzzing without consumer code}}
    \location{\href{https://conf.researchr.org/home/fse-2025}{FSE'25}}
    \authors{Flavio Toffalini, \textbf{Nicolas Badoux}, Zurab Tsinadze, \& Mathias Payer.}
    \description{%
    \vspace{-0.1cm}
      \item \'Ecrire des fuzz drivers, des séquences d'appel à une librairie pour du fuzzing, est complexe.
      % 
      \item \textsc{libErator} automatise leur création
      sans le besoin de code externe à la librairie et équilibre les ressources entre la création et le test des drivers.
       Via des passes LLVM, nous comprenons l'utilisation de la librairie et construisons des drivers C valides.
      Nous reportons 24 bugs, dont la
      \href{https://nvd.nist.gov/vuln/detail/cve-2024-8006}{CVE-2024-8006}. Notre prototype est sur \underline{\href{https://github.com/HexHive/liberator}{Github}}.
      %
      \item Pour l'évauation multifacette ainsi que le design de \textsc{libErator}, 
      j'ai du anticiper les complexités futurs et comprendre de manière transversale les caractéristiques des systèmes.
      %
    }
\end{research}
\begin{research}
    \title{\href{https://nebelwelt.net/files/25DIMVA.pdf}{Sourcerer: channeling the void}}
    \authors{\textbf{Nicolas Badoux}, Flavio Toffalini, \& Mathias Payer.}
    \location{\href{https://www.dimva.org/dimva2025/}{DIMVA'25}}
    \description{%
      \item En C++, les conversions entre \texttt{void*} et des pointers typés sont courantes mais,
      si le type de destination diffère de celui d'origine, peuvent amenés à la corruption de la mémoire.
      %
      \item En étendant la protection de type++ à tous les types, Sourcerer est le premier sanitizer complet pour ces erreurs. 
      Avec un ralentissement de seulement 5\% en moyenne, nous conduisons la première campagne de fuzzing visant spécifiquement les confusions de types.
      %
      \item Sourcerer est disponible sur 
      \href{https://github.com/HexHive/Sourcerer}{\underline{GitHub}} et trouve
      des erreurs dans Blender et OpenCV. 
      %
      \item Comme auteur principal, j'ai conçu l'architecture de Sourcerer, pris en charge l'évaluation et l'écriture de l'article.
      }
\end{research}
\begin{research}
    \authors{Nicolas Almerge, \textbf{Nicolas Badoux}, \& Mathias Payer.}
    \title{Bypassing LLVM-CFI cast protection }
    \location{En cours}
    \description{
      \item Notre nouvelle attaque permet de contourner les protections de conversions de LLVM-CFI.
      %
      \item Comme superviseur principal de ce projet de Master, j'ai défini le plan de recherche, guidé le travail, quantifé les résultats, ainsi qu'aidé à la rédaction du rapport.
    }
\end{research}
\vspace{-\parskip}
\sectionline
%-------------------------------------------------------------------------------
%-------------------------------------------------------------------------------
%	WORK SECTION
%-------------------------------------------------------------------------------
\section{\'Experience Industrielle}
\begin{position}
    \employer{\href{https://digger.ch}{Fondation Digger}\normalfont{, ONG}}
    \title{Ingénieur Informatique}
    \location{Tavannes, CH}
    \duration{Août '19--Mars '20}
    \description{
      \item Lors de mon service civil, développement, au sein d'un environement Agile, d'une surcouche visuelle  pour pouvoir détonner des mines avec une pelleteuse télécomandée.  
    }
\end{position}
\begin{position}
    \employer{\href{https://compassion.ch}{Compassion Suisse}\normalfont{, ONG}}
    \title{Ingénieur Informatique}
    \location{Yverdon, CH}
    \duration{Mars--Mai '18}
    \description{
        \item Comme service civil, j'ai contribué aux modules Python
        \href{https://github.com/CompassionCH/compassion-switzerland}{open source}
        pour l'\href{https://www.odoo.com/}{ERP Odoo}.
    }
\end{position}
\begin{position}
    \employer{\href{https://ergon.ch/en}{Ergon}}
    \title{Ingénieur Informatique Stagiaire}
    \location{Z\"urich, CH}
    \duration{60\%---Sept' '17--Mars '18}
    \description{
        \item Création d'un fuzzer blackbox en Python pour tester le
        \href{https://www.airlock.com/en/products/airlock-waf/}{Web Application
        Firewall}.
    }
\end{position}
\begin{position}
    \employer{\href{https://www.morganstanley.com/}{Morgan Stanley}}
    \title{Analyste Technologique Stagiaire}
    \location{London, UK}
    \duration{Juin--Août '16}
    \description{
        \item Dévelopement de tableaux statistiques en AngularJS pour l'équipe Sécurité.
    }
\end{position}
\vspace{-\parskip}
\sectionline

%-------------------------------------------------------------------------------
%  SKILLS SECTION
%-------------------------------------------------------------------------------
\section{Compétences}
\cusemph{Languages de programmation}: Python, C++, \LaTeX, Bash.

\cusemph{Logiciels}: LLVM, Docker, GDB, Linux, libfuzzer.

\cusemph{Langues}: Français (maternel), Anglais, Suisse-Allemand, Allemand.

\sectionline
%-------------------------------------------------------------------------------
%  TEACHING SECTION
%-------------------------------------------------------------------------------
\section{Assistanat}
%

\cusemph{CS-119 Information, calcul \& communication} \hfill\textit{ '22 \& '24}

\cusemph{CS-323 Systèmes d'opération} \hfill \textit{'21}

\cusemph{CS-412 Sécurité logicielle} \hfill \textit{'21 \& '23}

\cusemph{COM-402 Sécurité de l'information \& vie privée} \hfill \textit{'23}

\sectionline
%\section{Talks}
%
%\cusemph{NDSS'25} - \texttt{type++}: prohibiting type confusion with inline type information \timeline{2025}
%
%\sectionline
%-------------------------------------------------------------------------------

%-------------------------------------------------------------------------------
%	Interests
%-------------------------------------------------------------------------------
\section{Activités}
\begin{extra}
  \title{Membre du Conseil, Trésorier \textnormal{-
  \href{https://www.gbeu.ch}{Groupes Bibliques des Écoles et Universités}}}
  \location{'23--présent}
  \description{%
    \item Définition de la stratégie, participation au processus de recrutement, et planification budgétaire ($\simeq$ 500kCHF).
  }
\end{extra}
\begin{extra}
  \title{Directeur de Camp \textnormal{-
  \href{https://www.interjeunes.net/explosion-camp}{Interjeunes} \&
  \href{https://ligue.ch}{Ligue pour la Lecture de la Bible}}}
  \location{'14, '17, '21, '22}
  \description{
    \item Direction de plusieurs camps avec jusqu'à 110 enfants/jeunes adultes durant une semaine. Recrutement d'une équipe,
    préparation de l'événement, management de l'équipe, et en charge de l'autorité durant la semaine.
  }
\end{extra}

\sectionline
\section{Références}
\begin{extra}
  \title{\href{https://nebelwelt.net/}{Prof. Dr. Mathias Payer}}
  \location{\href{mailto:mathias.payer@nebelwelt.net}{mathias.payer@nebelwelt.net}}
  \description{%
    \item \href{https://nebelwelt.net/}{Professeur Associé à l'EPFL à Lausanne
    (CH)} et chef du laboratoire \href{https://www.hexhive.epfl.ch}{HexHive}.
    %
    \item Superviseur durant mon doctorat entre 2020 et 2025. 
  }
\end{extra}
\begin{extra}
  \title{\href{https://flaviotoffalini.info/}{Prof. Dr. Flavio Toffalini}}
  \location{\href{mailto:flavio.toffalini@rub.de}{flavio.toffalini@rub.de}}
  \description{
    \item \href{https://flaviotoffalini.info/}{Professeur Assistant à la Ruhr-Universität} Bochum (DE).
    \item Proche collaborateur et post-doc durant la majorité de mon doctorat (2021--2025). 
  }
\end{extra}
\begin{extra}
  \title{Benoît Pfister}
  \location{\href{mailto:benoit.pfister@gbeu.ch}{benoit.pfister@gbeu.ch}}
  \description{
    \item Président du Conseil des Groupes Bibliques des Écoles et Universités. 
    \item Nous avons travaillé ensemble dans des comités d'engagement, pour le budget et la stratégie générale.
  }
\end{extra}

\end{resume}
\end{document}
