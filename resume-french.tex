%%%%%%%%%%%%%%%%%%%%%%%%%%%%%%%%%%%%%%%%%%%%%%%%%%%%%%%%%%%%%%%%%%%%%%%%%%%%%%%%
% Medium Length Graduate Curriculum Vitae
% LaTeX Template
% Version 1.2 (3/28/15)
%
% This template has been downloaded from:
% http://www.LaTeXTemplates.com
%
% Original author:
% Rensselaer Polytechnic Institute 
% (http://www.rpi.edu/dept/arc/training/latex/resumes/)
%
% Modified by:
% Daniel L Marks <xleafr@gmail.com> 3/28/2015
% 
% Further modified by:
% Rohan Bavishi <rohan.bavishi95@gmail.com> 9/20/2016
%
% Further modified by:
% Nicolas Badoux <n.badoux@hotmail.com> 11/3/2018
%
% Important note:
% This template requires the simple_style.cls file to be in the same directory 
% as the .tex file. The res.cls file provides the resume style used for 
% structuring the document.
%
%%%%%%%%%%%%%%%%%%%%%%%%%%%%%%%%%%%%%%%%%%%%%%%%%%%%%%%%%%%%%%%%%%%%%%%%%%%%%%%%

%-------------------------------------------------------------------------------
%	PACKAGES AND OTHER DOCUMENT CONFIGURATIONS
%-------------------------------------------------------------------------------

%%%%%%%%%%%%%%%%%%%%%%%%%%%%%%%%%%%%%%%%%%%%%%%%%%%%%%%%%%%%%%%%%%%%%%%%%%%%%%%%
% You can have multiple style options the legal options ones are:
%
%   centered:	the name and address are centered at the top of the page 
%				(default)
%
%   line:		the name is the left with a horizontal line then the address to
%				the right
%
%   overlapped:	the section titles overlap the body text (default)
%
%   margin:		the section titles are to the left of the body text
%		
%   11pt:		use 11 point fonts instead of 10 point fonts
%
%   12pt:		use 12 point fonts instead of 10 point fonts
%
%%%%%%%%%%%%%%%%%%%%%%%%%%%%%%%%%%%%%%%%%%%%%%%%%%%%%%%%%%%%%%%%%%%%%%%%%%%%%%%%

\documentclass[mm, 11pt]{simple_style}  

% Default font is the helvetica postscript font
\usepackage{helvet}
\usepackage{hyperref}
\usepackage{url}
\usepackage{xcolor}
\hypersetup {
    colorlinks=false,
    linkcolor=colorlink,
    filecolor=magenta,      
    urlcolor=colorlink,
}
\usepackage[a4paper, left=12mm, right=45mm, top=15mm, bottom=15mm]{geometry}
\usepackage[fristpage=true, color=black, opacity=0.5, angle=90]{background}
\usepackage[french]{babel}

\begin{document}

\SetBgContents{Version du \today}% Set contents
\SetBgPosition{ \textwidth / 0.8 - 0.8cm,-24cm/0.8}% Select location
\SetBgScale{0.8}% Select scale factor of logo

\newsectionwidth{32mm}
%-------------------------------------------------------------------------------
%	NAME AND ADDRESS SECTION
%-------------------------------------------------------------------------------
\name{Nicolas Badoux}
%\qualification{\'Etudiant master en S\'ecurit\'e informatique}
\emailone{n.badoux@hotmail.com}
%\emailtwo{badouxn@ethz.ch}
%\website{https://nicolasbadoux.com}{\url{nicolasbadoux.com}}
%\github{https://github.com/vwvw}{\url{github.com/vwvw}}
\phone{+41 79 914 00 47}
\citizenship{Citoyen Suisse}
\birthdate{n\'e le 06.11.1994}

\address{\href{https://goo.gl/maps/bxwsN3DiRHF2}{Grand-Rue 8}
    \\\href{https://goo.gl/maps/bxwsN3DiRHF2}{1315} \href{https://goo.gl/maps/bxwsN3DiRHF2}{La Sarraz}
    \\\href{https://goo.gl/maps/bxwsN3DiRHF2}{CH - Switzerland}
}

\begin{resume}
%-------------------------------------------------------------------------------
\section{\'Education}
\cusemph{\href{https://www.inf.ethz.ch/}{Eidgen\"ossische Technische Hochschule Z\"urich (ETHZ)}} - Suisse \timeline{2016-2019}\\
{\sl Master of Science ETH in Computer Science} (anticip\'e)\\
Sp\'ecialisation en s\'ecurit\'e informatique\\
\cusemph{\href{https://www.ece.cmu.edu/}{Carnegie Mellon University (CMU)}} - Pittsburgh PA - USA \timeline{2015-2016}\\
Programme d'\'echange\\
\cusemph{\href{https://ic.epfl.ch/en}{\'Ecole Polytechnique F\'ed\'erale de Lausanne (EPFL)}} - Suisse  \timeline{2013-2016}\\
{\sl Bachelor en Syst\`emes de communication} - Moyenne g\'en\'erale: 5.26 / 6\\
\sectionline
%-------------------------------------------------------------------------------
%	RESEARCH SECTION
%-------------------------------------------------------------------------------
\section{Exp\'erience\\professionelle}
\begin{position}
    \employer{\href{https://chainsolutions.com/}{chainSolutions}}
    \title{Blockchain Consultant}
    \location{Z\"urich}
    \duration{Jan' - Juin 2018 @ 20\%}
    \description{
        -\,Design d'architecture de diff\'erents services pour des clients avec un focus sur l'identit\'e digitiale decentralis\'ee et la confidentialit\'e}
\end{position}\\
\begin{position}
    \employer{\href{https://compassion.ch}{Compassion Suisse}, ONG}
    \title{D\'eveloppeur (Service Civil)}
    \location{Yverdon}
    \duration{Mars - Mai 2018}
    \description{
        -\,Ajout de module en Python à l'ERP \href{https://www.odoo.com/}{Odoo}. Fonctionnement Agile et travaille \href{https://github.com/CompassionCH/compassion-switzerland}{open-source}.
    }
\end{position}\\
\begin{position}
    \employer{\href{https://ergon.ch/en}{Ergon Informatik}}
    \title{Security Engineer stagiaire}
    \location{Z\"urich}
    \duration{Sept' 2017 - Mars 2018 @ 60\%}
    \description{
        -\,D\'eveloppement, en Python, d'un outil de blackbox fuzzing pour am\'eliorer la s\'ecurit\'e du \href{https://www.airlock.com/en/products/airlock-waf/}{Web Application Firewall}. Analyse du support des requ\^etes HTTP Multipart. D\'etection de plusieurs erreurs logiques qui auraient pu servir \`a contourner le WAF.
    }
\end{position}\\
\begin{position}
    \employer{\href{https://www.morganstanley.com/}{Morgan Stanley}}
    \title{Technology Summer Analyst}
    \location{Londres, UK}
    \duration{Juin - Ao\^ut 2016}
    \description{
        -\,D\'eveloppement de graphiques personnalisables pour analyser les statistiques au sein de l'\'equipe Architecture Security.\par
        -\,Identification des graphiques pertinents et des besoins du management. Impl\'ementation en combinant deux sources de donn\'ees diff\'erentes.\par
        -\,Backend construit en Java avec myBatis, frontend en AngularJS et Highcharts.
    }
\end{position}\\
\sectionline
\section{Cours suivis}
    \begin{tabular}{lllll}
        \href{http://www.syssec.ethz.ch/education/sown/sown_AS16.html}{Security of Wireless Network} & \href{http://www.syssec.ethz.ch/education/system_security/system_security_as16.html}{System Security}\\
        \href{https://netsec.ethz.ch/courses/netsec-2016/}{Network Security} & \href{http://www.infsec.ethz.ch/education/as2016.html}{Applied Security Lab} \\
        \href{https://www.cs.cmu.edu/~10601b/}{Introduction to Machine Learning} & \href{https://ndal.ethz.ch/courses/acn.html}{Advanced Computer Networks}\\
    \end{tabular}\\
\sectionline
%-------------------------------------------------------------------------------
%	SKILLS SECTION
%-------------------------------------------------------------------------------
\section{Comp\'etences}
\cusemph{Languages de programmation}: Python, Java, Matlab, Solidity, \LaTeX\\
\cusemph{Langues parl\'ees}: Fran\c{c}ais (maternelle), anglais, suisse-allemand, allemand\\
\sectionline
%-------------------------------------------------------------------------------
%  PROJECTS SECTION
%-------------------------------------------------------------------------------
\section{Projets}
\begin{project}
  \title{Am\'elioration de \href{https://github.com/HexHive/T-Fuzz}{T-Fuzz} \`a l'aide d'\'elagage de code}
  \duration{Sept' 2018 - Actuel}
  \location{EPFL}
  \description{
	-\,Augmentaiton des capacit\'es du mutateur de programe pour enelever des voies ch\`eres computationellement ou inutiles.\par
	-\,Th\`ese de master sous la supervsion de \href{https://nebelwelt.net/}{Prof. Mathias Payer}, \'ecrit en Python.
  }
\end{project}\\
\begin{project}
  \title{\href{https://www.ethz.ch/en/the-eth-zurich/global/global-network/eth-studios/new-york-security-challenge-2018.html}{Bloomberg New York Security Challenge}}
  \duration{Juin 2018}
  \location{NY}
  \description{
    - Comme l'un des gagnants de l'\href{https://blogs.ethz.ch/ETHambassadors/2018/07/26/coding-challenge-inside-bloomberg/}{ETH Codecon}, participation à un projet d'une semaine au sein du \href{https://www.bloomberg.com/careers/technology/}{CTO Office} consistant à l'analyse et la d\'etection d'anomalies dans les traces r\'eseaux de microservices.
  }
\end{project}\\
\begin{project}
  \title{\href{https://github.com/vwvw/Ethereum-RingCT}{RingCT dans un smart contract Ethereum}}
  \duration{Mars - Sept' 2017}
  \location{ETHZ}
  \description{
	-\,Impl\'ementation en Solidity d’un smart contract offrant des paiements anonymes similaire \`a ceux de Monero.\par
	-\,Projet effectu\'e comme introduction \`a la recherche et supervis\'e par \href{http://arthurgervais.com/}{Prof. Arthur Gervais}.
  }
\end{project}\\
\begin{project}
  \title{\href{https://github.com/vwvw/Greenheater}{Greenheater - un chauffage \'electrique intelligent}}
  \duration{Jan' - Mai 2016}
  \location{CMU}
  \description{
	-\,Automatisation d’un chauffage \'electrique d’appoint profitant des p\'eriodes d’absences pour \'economiser de l’\'energie sans impacter le confort.\par
	-\,\'Ecrit en Python avec l’apprentissage automatique effectu\'e sur Matlab.
  }
\end{project}\\
\sectionline
%-------------------------------------------------------------------------------

%-------------------------------------------------------------------------------
%	Interests
%-------------------------------------------------------------------------------
\section{Activit\'es}
\cusemph{\href{https://www.interjeunes.net/explosion-camp/}{Directeur de camp de ski}}, Grimentz, \timeline{2014 \& 2017}\\
Direction et planification d’un camp avec 110 adolescents et jeunes adultes durant une semaine. Gestion d'\'equipe de responsables et du bon d\'eroulement du camp.\\
\cusemph{Responsable de \href{http://www.gbeu.ch/}{Groupe Biblique}}, EPFL, \timeline{2014-2015}\\
Responsable d'un groupe de discussion autour de la Bible consitu\'e d'\'etudiants de l'EPFL.


%-------------------------------------------------------------------------------
\end{resume}
\end{document}